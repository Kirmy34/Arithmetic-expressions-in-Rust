\documentclass[a4paper, 1ppt]{article}
\usepackage{graphicx} % Required for inserting images
\usepackage{listings}
\usepackage{abstract}
\usepackage{url}
\usepackage[german]{babel}
\usepackage{hyperref}
\usepackage{hyphenat}
\usepackage{minted}
\clubpenalty=10000
\widowpenalty=1000
\title{Arithmetische Ausdrücke in Rust - Dokumentation}
\date{}
\author{
		Mario Occhinegro (74661)\\
		Michael Kirmizakis (75592)\\
		Jonas Heck (67889)
}
\usepackage{setspace}
\onehalfspacing
\singlespacing
\begin{document}
\nocite{*}
\pagenumbering{gobble} 
\maketitle
\newpage
\clearpage
\tableofcontents
\setcounter{page}{1}
\newpage
\pagenumbering{arabic}
\maketitle
\section{Datatype}
Einfaches Enum zur Bestimmung der Art des Datentyps.
Wir haben Integer oder boolsche Werte
\section{Expression Enum}
Das Expression Enum beschreibt alle möglichen Ausdrücke, die wir verarbeiten können.
Eine Variante jeweils für die Zahlen von 1-9. Sowie jeweils eine Variante für true und false.
Für die Operatoren Plus, Mult, Or und And gibt auch jeweils eine Variante. Interessant ist hier, dass 
diese hier wieder Unterexpressionen verwalten. Ein mal für Rechts und ein mal für Links.
\section{Evaluieren von Expressions}
Um eine Expression zu evaluieren, wird das Pattern-Matching von Rust verwendet. Bezüglich des Rückgabetyps, ist zu beachten, dass aufgrund der Tatsache, dass in den Ausdrücken sowohl logische Ausdrücke als auch Ausdrücke mit Zahlen vorkommen, hier eine Unterscheidung benötigt wird. Weiterhin besteht die Möglichkeit, dass ein Ausdruck gar nicht evaluiert werden, kann, weil eventuell eine nicht auflösbare Vermischung von logischen und mathematischen Ausdrücken vorkommt. Manche dieser vermischten Ausdrücke können jedoch aufgrund der Short-Circuit-Evaluation trotzdem ausgewertet werden.
Um nun den Rückgabewert der Evaluate-Funktion zu definieren, wird Option<Result<i32, bool>> gewählt, da mittels der Option, angegeben werden kann, ob überhaupt ein Wert zurückgegeben wird (Some), oder ob die Auswertung gescheitert ist und kein Wert zurückgegeben werden kann (None). Um zwischen einer Zahl und einem logischen Wert zu unterscheiden, wird Result<i32, bool> verwendet, hierdurch, kann mittels Ok(number) eine Zahl und mittels Err(logic) ein logischer Wert zurückgegeben werden, dies wird für das Pattern-Matching benötigt, da es hierdurch einfacher wird, den Typ einer Evaluation zu matchen und zu erkennen.
Um jetzt die eigentliche Evaluation durchzuführen, werden zunächst die Patterns, für Expression angegeben, welche nur eine Zahl enthalten. Das bedeutet, es werden die Expression Zero bis Nine im Pattern-Matching angegeben und dann die entsprechende Zahl zurückgegeben, sowie die Patterns für die booleschen Werte true und false.
Danach folgt das Pattern für eine Plus-Expression. Hier werden zunächst, beide Seiten der Plus-Expression evaluiert. Danach folgt ein Pattern-Matching auf die beiden Ergebnisse, wenn beide Ergebnisse einen Zahlen-Wert liefern (Some(Ok(left\
\_value))) werden die beiden Werte von der linken und rechten Seite aufaddiert und an die aufrufende Funktion zurückgegeben. Wenn eine der beiden Seiten keine Zahl zurückliefert, wird dies mittels des don't care Patterns \_ im entsprechenden Fall gefangen und None zurückgegeben, um anzuzeigen, dass keine Auswertung der Expression möglich war. Hier wäre eine kleine weitere mögliche Verbesserung, zunächst nur den linken Wert zu evaluieren und den rechten Teil der Expression nur zu evaluieren, falls im linken Teil auch tatsächlich eine Zahl steht. Falls im linken Teil schon keine Zahl steht, könnte man die Evaluation der Expression hier direkt abbrechen und Nonen zurückgeben und sich hierdurch die Evaluation der rechten Seite sparen.
Als nächstes erfolgt die Pattern Prüfung, auf eine Multiplikation-Expression. Hier wird zunächst, geprüft, ob an der linken Stelle der Expression eine 0 steht, in diesem Fall, kann die weitere Evaluierung übersprungen werden und direkt 0 zurückgegeben werden, da 0 * \_ immer 0 ist. In allen anderen möglichen Fällen, wird dann die linke Seite, der Expression evaluiert. Danach wird auf das Ergebnis dieser Evaluation ein Pattern Matching angewandt, falls die Evaluierung der linken Seite der Expression ergab, dass diese 0 ist, wird direkt 0 zurückgeben. Wenn die linke Seite zu einer Zahl ungleich 0 evaluiert wurde, wird auch die rechte Seite evaluiert. Für den Fall, dass die linke Seite gar nicht evaluiert werden konnte, wird mittels des don't care Patterns None zurückgegeben. Nachdem nun die rechte Seite evaluiert wurde, wird auch auf diesem Ergebnis ein Pattern-Matching durchgeführt, was dieselben Fälle wie das Pattern Matching für die linke Seite abdeckt. Lediglich, für den Fall, dass ein Zahlen-Wert ungleich 0 ermittelt wurde, für die rechte Seite, wird anders als im Fall für die linke Seite, die Zahlen der linken und rechten Seite addiert und zurückgegeben.
Um eine Oder Expression auszuwerten, wird zunächst die linke Seite auswertet und dann eine Pattern-Matching auf diesem Ergebnis gemacht. Falls die linke Seite nicht ausgewertet werden konnte, wird direkt None zurückgegeben. Wenn im Rahmen der Auswertung der linken Seite, eine Zahl ermittelt wurde, wird die Auswertung ebenfalls abgebrochen und None zurückgegeben. Wenn die linke Seite jedoch zu einem logischen Wert ausgewertet werden konnte, wird auf diesem Wert wiederum ein Pattern Matching gestartet, da für den Fall, dass die linke Seite der Oder Expression zu true ausgewertet wurde, direkt true als Wert für die Expression zurückgegeben werden kann, da die linke Seite dann das Ergebnis der Auswertung nicht mehr verändern kann. Für den Fall, dass die linke Seite zu false ausgewertet wurde, wird auch die rechte Seite ausgewertet und dann auch auf diesem Ergebnis, ein Pattern Matching ausgeführt.
Im Rahmen dieses Pattern Matching wird dann auch hier wie bei der linken Seite geprüft, ob die Expression überhaupt evaluiert werden konnte oder und ob überhaupt ein boolescher Wert für die rechte Seite evaluiert wurde. Falls ein boolescher Wert auch auf der rechten Seite steht, wird einfach dieser als Wert der Expression zurückgegeben, da der Wert der Expression nun nur noch von diesem abhängt.
Abschließend folgt noch das Pattern für die und Expression. Bei diesem wird ebenfalls zuerst die linke Seite der Expression ausgewertet und auf dem Ergebnis dieser Auswertung ein Pattern-Matching durchgeführt. Falls die linke Seite nicht evaluiert werden konnte, wird None zurückgegeben. Falls die linke Seite zu einem Zahlen-Wert evaluiert wurde, wird die weitere Auswertung abgebrochen und None zurückgegeben. Nur wenn die linke Seite auch zu einem booleschen Wert ausgewertet wurde, wird auf diesem Ergebnis wiederum ein Pattern Matching ausgeführt, mittels welchem zunächst geprüft wird, ob der ermittelte Wert false ist, in diesem Fall, wird die rechte Seite gar nicht ausgewertet, da bei einer und Expression sowohl die rechte als auch die linke Seite true sein müssen und direkt false zurückgegeben. Wenn der linke Teil der und Expression true ist, wird auch der rechte ausgewertet und geprüft, dass die Auswertung einen Wahrheitswert ergeben hat, ansonsten wird direkt None zurückgegeben, als Indikator, dass die Auswertung fehlschlug. Wenn die Auswertung der rechten Seite, einen booleschen Wert ergibt, wird einfach dieser zurückgegeben, da dieser Wert das Ergebnis der Expression bestimmt. Dies ist der Fall, da der linke Teil zu diesem Punkt im Code schon zu true ausgewertet wurde.
\section{Parsen von Expressions}
Die Funktion, die das Parsen von Expression zuständig ist, funktioniert wie folgt.
Sie bekommt einen String und ein Präzedenzlevel, um sich in der Rekursion zu merken
was die Präzedenz der Parent-Expression ist. Die Ausgabe ist dann ein Expression-Enum, das die eingegebene Expression beschreibt.
Innerhalb des Rumpfes bewegen wir uns von links nach rechts durch den übergebenen String.
Jeder Aufruf fängt damit an, dass wir den Charakter auf dem wir uns befinden parsen.
Dies ist immer eine Zahl, ein Boolean oder ein geklammerter Ausdruck. 
Diese Expression merken wir uns als linken Teilbaum.
Danach betrachten wir, falls vorhanden, den nächsten Charakter, bei dem es sich in unserer Logik um einen Operator handeln muss, 
da wir keine unnären Operatoren unterstützen. Falls nicht vorhanden, geben wir die linke Seite als Expression zurück und sind fertig.
Für den Fall, dass es sich aber wirklich um einen Operator handelt, bestimmen wir zu aller erst dessen Präzedenz.
Falls die übergeordnete Präzedenz der Parent Expression höher ist, so brechen wir an dieser Stelle ab und geben den bis dato errechnetem Baum zurück.
Das ist notwendig, da falls die Präzedenz des übergeordneten Operators größer ist, der momentane Charakter teil der Parent-Expression ist.
Falls für den anderen Fall setzen wir den Operator in die Wurzel unseres Baums. 
Jetzt fehlt nur noch die rechte Seite unseres Baumes.
Diese berechnen wir rekursiv auf dem Rest des Iterables und dem momentanen Präzedenz-Operator.
Falls der Unteraufruf den String nicht voll aufbraucht, springen wir wieder in While-Schleife der höheren Rekursionsebene.
Jetzt updaten wir den linken Teilbaum mit dem berechneten Ergebnis.
Dies tun wir so lange, bis keine Characters mehr übrig sind.
\section{Expression Typcheck}
Diese Methode prüft den Datentyp eines Ausdrucks und gibt ein Ergebnis zurück.
implementiert die methode typecheck für den expressiontyp.
wenn der ausdruck eine zahl zwischen 0 und 9 ist, wird der datentyp tint zurückgegeben.
wenn der ausdruck etrue oder efalse ist, wird der datentyp tbool zurückgegeben.
wenn der ausdruck eine addition, multiplikation, logische oder oder logische undoperation ist, werden die datentypen der linken und rechten operanden überprüft.
Wenn beide Operanden den richtigen Typ haben, wird TInt bzw.
TBool zurückgegeben, andernfalls wird None zurückgegeben.
\end{document}
