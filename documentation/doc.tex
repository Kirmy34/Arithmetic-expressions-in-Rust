\documentclass[a4paper, 1ppt]{article}
\usepackage{graphicx} % Required for inserting images
\usepackage{listings}
\usepackage{abstract}
\usepackage{url}
\usepackage[german]{babel}
\usepackage{hyperref}
\usepackage{hyphenat}
\usepackage{minted}
\clubpenalty=10000
\widowpenalty=1000
\title{Arithmetische Ausdrücke in Rust - Dokumentation}
\date{}
\author{
		Mario Occhinegro (74661)\\
		Michael Kirmizakis (75592)\\
		Vorname Nachname ()\\
}
\usepackage{setspace}
\onehalfspacing
\singlespacing
\begin{document}
\nocite{*}
\pagenumbering{gobble} 
\maketitle
\newpage
\clearpage
\tableofcontents
\setcounter{page}{1}
\newpage
\pagenumbering{arabic}
\maketitle
\section{Datatype}
\subsection{Datatyp Funktionen}
\section{Expression Enum}
\section{Evaluieren von Expressions}
\section{Parsen von Expressions}
\section{Expression Typcheck}
Implementiert die Methode type-check für den Expression-Typ.
Diese Methode prüft den Datentyp eines Ausdrucks und gibt ein Ergebnis zurück.
Wenn der Ausdruck eine Zahl zwischen 0 und 9 ist, wird der Datentyp TInt zurückgegeben.
Wenn der Ausdruck ETrue oder EFalse ist, wird der Datentyp TBool zurückgegeben.
Wenn der Ausdruck eine Addition, Multiplikation, logische ODER- oder logische UND-Operation ist, werden die Datentypen der linken und rechten Operanden überprüft.
Wenn beide Operanden den richtigen Typ haben, wird TInt bzw.
TBool zurückgegeben, andernfalls wird None zurückgegeben.
Der Code enthält auch eine Testfunktion, die verschiedene Ausdrücke initialisiert und die type-check-Methode auf sie anwendet, um die Funktionalität zu überprüfen.
\end{document}
